\documentclass{article}[11 pt]
\usepackage{graphics}
\usepackage{epsfig}
\usepackage{color}
\usepackage{graphicx,psfrag,pst-node,subfigure,verbatim}
\usepackage{amsmath,bbm,amssymb,mathrsfs}
\usepackage{amsthm}
\usepackage{comment}
\usepackage{rotating}
\usepackage{lscape}
\usepackage{enumerate}
\usepackage{fullpage}
\usepackage{url}
\begin{document}
\begin{center}	
	\textbf{Mexican Drug War: Have the Military Interventions Increased Violence?}
\end{center}

When we start to approach this problem the first to consider is what data is available to analyze it. We are using the Military interventions listed in a non comprehensive list from may 7,2007 - nov 21, 2010) NEXOS paper: \url{http://www.nexos.com.mx/?P=leerarticulo&Article=1943189 }. %Read this again to use their terminology.

The measure of violence that we are using is homicide rate, Y. This could be debatable, maybe a combined measure of different kinds of crimes would be better.

Assumptions:
\begin{itemize}
	\item SUTVA
	\item Unconfoundedness: a very strong assumption is that we have all the covariates that could affect the homicide rates. In other words that we have all covariates, X, such that given X,  W is independent of Y.
	\item That homicide rates, Y, are an accurate measure of violence. 
\end{itemize}

The main idea is to combine synthetic and propensity score matching to address this question. What are the advantages of doing that? (and what's new?)
The current proposal is to use propensity score matching to create pools of acceptable controls for each treated unit (is there some multiple comparison thing going on here? Probably not, we just cared about the observed imbalances, we are not saying anything else.). Furthermore, to get the synthetic match for treated unit $T_i$ we can choose the weights for the units in it's control pool such that the $Y_1^T,\ldots, Y_{I_i}^{T_i}$ is best matched
	

There are 2213 municipalities in the initial control pool, and the numbers per treated unit are:
  % latex table generated in R 2.14.0 by xtable 1.6-0 package
% Thu Sep 27 10:40:57 2012
% \begin{table}[ht]
% \begin{center}
% \begin{tabular}{rc|rc}
% 
%   \hline 
%   \textbf{1 }&   5 &\textbf{10} &  10 \\ 
%   \textbf{2} &   5 &\textbf{11} &   8 \\  
%   \textbf{3} &  12 &\textbf{12 }&  27 \\ 
%   \textbf{4} &  15 & \textbf{13} &  11 \\
%   \textbf{5} &  14 & \textbf{14} &   9 \\ 
%   \textbf{6} &  24 & \textbf{15} &   9 \\ 
%   \textbf{7} &   5 &  \textbf{16} &  10 \\ 
%   \textbf{8} &  20 &\textbf{17} &   6 \\ 
%   \textbf{9} &  18 & \textbf{18 }&  35 \\
% \hline
% \end{tabular}
% \end{center}
% \end{table}


% latex table generated in R 2.14.0 by xtable 1.6-0 package
% Thu Sep 27 14:57:21 2012
\begin{table}[ht]
\begin{center}
\begin{tabular}{ccc|ccc}
  \hline
 \textbf{unit}& number of& Date of first &  \textbf{unit}& number of& Date of first \\ 
 & municipalities& intervention& municipalities& intervention \\ 
  \hline
\textbf{1} &   5 & 2008 &  \textbf{10} &10 & 2009 \\ 
  \textbf{2} &   5 & 2008 &  \textbf{11} & 8 & 2008 \\ 
  \textbf{3} &  12 & 2010 &  \textbf{12} &27 & 2007 \\ 
  \textbf{4} &  15 & 2009 &  \textbf{13} &11 & 2010 \\ 
  \textbf{5} &  14 & 2007 &   \textbf{14} &9 & 2010 \\ 
  \textbf{6} &  24 & 2008 &   \textbf{15} &9 & 2009 \\ 
  \textbf{7} &   5 & 2010 &  \textbf{16} &10 & 2007 \\ 
  \textbf{8} &  20 & 2009 &   \textbf{17} &6 & 2010 \\ 
  \textbf{9} &  18 & 2008 &  \textbf{18} &35 & 2008 \\ 
   \hline
\end{tabular}
\end{center}
\end{table}
	
\end{document}