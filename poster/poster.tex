
%%%%%%%%%%%%%%%%%%%%%%%%%%%%%%%%%%%%%%
% LaTeX poster template
% Created by Nathaniel Johnston
% August 2009
% http://www.nathanieljohnston.com/2009/08/latex-poster-template/
%%%%%%%%%%%%%%%%%%%%%%%%%%%%%%%%%%%%%%

\documentclass[final]{beamer}
\usepackage[scale=1.24]{beamerposter}
\usepackage{graphicx}			% allows us to import images

%-----------------------------------------------------------
% Define the column width and poster size
% To set effective sepwid, onecolwid and twocolwid values, first choose how many columns you want and how much separation you want between columns
% The separation I chose is 0.024 and I want 4 columns
% Then set onecolwid to be (1-(4+1)*0.024)/4 = 0.22
% Set twocolwid to be 2*onecolwid + sepwid = 0.464
%-----------------------------------------------------------

\newlength{\sepwid}
\newlength{\onecolwid}
\newlength{\twocolwid}
\newlength{\threecolwid}
\setlength{\paperwidth}{48in}
\setlength{\paperheight}{36in}
\setlength{\sepwid}{0.024\paperwidth}
\setlength{\onecolwid}{0.22\paperwidth}
\setlength{\twocolwid}{0.464\paperwidth}
\setlength{\threecolwid}{0.708\paperwidth}
\setlength{\topmargin}{-0.5in}
\usetheme{confposter}
\usepackage{exscale}
\usepackage{calc}
\usepackage{caption}
\usepackage{textpos}
\usepackage{subfigure}

%-----------------------------------------------------------
% The next part fixes a problem with figure numbering. Thanks Nishan!
% When including a figure in your poster, be sure that the commands are typed in the following order:
% \begin{figure}
% \includegraphics[...]{...}
% \caption{...}
% \end{figure}
% That is, put the \caption after the \includegraphics
%-----------------------------------------------------------

\usecaptiontemplate{
\small
\structure{\insertcaptionname~\insertcaptionnumber:}
\insertcaption}

%-----------------------------------------------------------
% Define colours (see beamerthemeconfposter.sty to change these colour definitions)
%-----------------------------------------------------------

\setbeamercolor{block title}{fg=ngreen,bg=white}
\setbeamercolor{block body}{fg=black,bg=white}
\setbeamercolor{block alerted title}{fg=white,bg=dblue!70}
\setbeamercolor{block alerted body}{fg=black,bg=dblue!10}

%-----------------------------------------------------------
% Name and authors of poster/paper/research
%-----------------------------------------------------------

\title{\begin{textblock}{1}(0,-0.3)
\includegraphics[scale=1]{RevisedPlainLogo.jpg}
\end{textblock}Visualization and Causal Inference of the Mexican Drug War}
\author{Valeria Espinosa and Joseph Kelly \vskip0.5ex \small{vespinos@fas.harvard.edu, kelly2@fas.harvard.edu}}
\institute{Statistics Department, Harvard University}%\\[\medskipamount]


% \title{
% Visualization and Causal Inference of the Mexican Drug War
% }
% \author{Valeria Espinosa and Joseph Kelly \vskip0.5ex \small{vespinos@fas.harvard.edu, kelly2@fas.harvard.edu}}
% \institute{Department of Statistics, Harvard University}


%-----------------------------------------------------------
% Start the poster itself
%-----------------------------------------------------------

\begin{document}
\begin{frame}[t]
  \begin{columns}[t]											
    \begin{column}{\sepwid}\end{column}			% empty spacer column
    \begin{column}{\onecolwid}
 \vskip-3ex
      \begin{block}{Problem Description}
        The presidency of Felipe Calder\'{o}n (2006-2012) has been characterized for the war against organized crime, raising many questions regarding security and violence. We attempt to visualize and analyze homicide rates at the municipality level, to answer whether \textbf{homicide rates increase significantly after a military intervention}. Visually we link this to information obtained about the association of drug cartels to municipalities.
      \end{block}
      
      \begin{block}{Estimand}
        Let $Y_i(1)$ denote the homicide rate change in region $i$ from one year before to one year after receiving a military intervention, and $Y_i(0)$ what it would have been if it hadn't received it (Rubin Causal Model). Our estimand is the average causal effect of the military intervention, for the regions that could be intervened, $$\tau=\overline{Y}(1)-\overline{Y}(0)=\frac{ \sum_{i=1}^{N} Y_i(1)-Y_i(0)}{N},$$
where $N$ is the total number of such regions.
        Let $N_i$ denote the number of municipalities that correspond to region $i$, then 
	$$Y_i(1) = \sum_{j=1}^{N_i}w_{ij}Y_{ij}(1) \textrm{ and } Y_i(0) = \sum_{j=1}^{N_i}w_{ij}Y_{ij}(0),$$	
	$$\textrm{ where }  w_{ij}= \frac{\textrm{Pop}_{ij}}{\textrm{Pop}_{i}} \textrm{ and  }\textrm{Pop}_{i}= \sum_j^{N_i}\textrm{Pop}_{ij}.$$
	%However, $Y_i(0)$ and $Y_{ij}(0)$ are missing  $\forall i, j$.
      \end{block}
      \begin{block}{Key Assumptions}
        {\small 
        \begin{itemize}
          \item \textbf{SUTVA: No hidden values of treatments}\\
            Broad definition of treatment: at least one municipality in the region received an intervention between 2007-2010, or not (\cite{NEXOS}).
          \item \textbf{SUTVA:No interference between units} \\
             Grouped close regions that received an intervention, and their neighboring municipalities to make the ``no interference'' assumption  more reasonable. %For treated regions that are side to side were also assessed in terms of neighboring  geographic situation such as lack of highways connecting them %\url{http://mx.kalipedia.com/kalipediamedia/geografia/media/200805/11/geomexico/20080511klpgeogmx_4_Ges_SCO.png} or big mountains between them, or closeness to where the crops or smuggling routes are \url{http://utopiaguatemala.files.wordpress.com/2011/11/mexico_routes_in_466.gif} (can we find such data?).
	%The last homicide rate that we have corresponds to 2010. That eliminated some of the interventions mentioned in the Nexos paper. }
        \item \textbf{Unconfoundedness} \\
           We assume we have all covariates, \textbf{X}, such that given \textbf{X},  treatment assignment is independent of \textbf{Y}.
          % Unfortunately we didn't get experts to guide most of our decisions. However, we did get to interact with a couple of them and made our covariate choices based on the information received and our understanding of the relevant information. 
          % Our covariates include: location, political party before Calder\'{o}n, income 2006, education and medical information at 2005, percentage of indigenous language speakers, 2006 homicide rate at the municipality level, and GDP, Homicide Rate and Population at the state level.
        \item \textbf{Missing Data}\\
           One covariate had missing values. We exactly matched on missingness pattern and Political Party in municipality before Calder\'{o}n. 
        \item \textbf{Appropriateness of response variable} We assume Y is an adequate measure of violence.	 %Perhaps other crimes should be included			
        \end{itemize}
      }
      \end{block}

    \end{column}

    \begin{column}{\sepwid}\end{column}			% empty spacer column
    \begin{column}{\twocolwid}
      \vskip-3ex 			
      \begin{block}{Estimation}
        The control pool consists of 2213 municipalities. Let I denote the number of treated regions (here 13, 205 municipalities).
          Let $M_{ij}$ be the number of municipalities matched, using propensity score, to the $j$th municipality in region $i$, and estimate $Y_{ij}(0)$ and $Y_i(0)$. Here $M_{ij}=5$.
        Let $\textrm{PopM}_{ij}=\sum_{k=1}^{M_{ij}}\textrm{PopM}_{ijk}$ be the sum of the populations of these matched municipalities. Then,
        $$\hat{Y}_{ij}(0) =\sum_{k=1}^{M_{ij}}v_{ijk}Y^{obs}_{ijk}(0),\quad \quad \textrm{ where } v_{ijk}=\frac{\textrm{PopM}_{ijk}}{\textrm{PopM}_{ij}}.$$
        Therefore,
        $$\hat{Y}_{i}(0) \quad =\quad \sum_{j=1}^{N_i}w_{ij}\hat{Y^{obs]}}_{ij}(0)\quad =\quad\sum_{j=1}^{N_i}w_{ij}\sum_{k=1}^{M_{ij}}v_{ijk}Y^{obs}_{ijk}(0)\quad =\quad
        \sum_{j=1}^{N_i}\sum_{k=1}^{M_{ij}}\tilde{w}_{ijk}Y^{obs}_{ijk}(0) \quad \quad \textrm{ with } \tilde{w}_{ijk} = w_{ij}v_{ijk}, $$
        and
        $$\hat{\tau}=\frac{\sum_i^{I} Y_i(1)}{I}-\frac{\sum_{i=1}^{I}\hat{Y}_i(0)}{I}=\overline{Y^{obs}}(1)-\overline{\hat{Y}}(0) .$$
        We know that $var(\hat{\tau})\leq var(\overline{Y}(1))+var(\overline{Y}(0))$,% add reference
  and it achieves the bound under additivity of potential outcomes. We use estimates of these quantities to estimate the variance. Note that the within region estimate of the variance is calculated based on the 5 synthetic control regions matched to each treated one.

% and assume the homicide rates have a Poisson distribution to get confidence intervals.% note that this is a case when we wouldn't want to be conservative!!!
        % Now, \begin{eqnarray*}
        %   var(\hat{\overline{Y}}(0))&=&E(var(\hat{\overline{Y}}(0)|Y_i(0) \forall i))+var(E(\hat{\overline{Y}}(0)|Y_i(0) \forall i))
        %   =E(\sum var(\hat{Y}_i(0))/I)+var(\sum_iY_i(0))/I\\
        %   &=&E(\frac{\sum_{j,k}w_{ijk}(Y_{ijk}(0)-Y_i(0))^2}{1-\sum_{j,k}w^2_{ijk}})+var(Y(0))/I =\frac{\sum_{i,j,k}w_{ijk}(Y_{ijk}(0)-Y_i(0))^2}{I(1-\sum_{j,w}w^2_{ijk})})+S^2(0)/I.\\
        % \end{eqnarray*}
        % Now, $var(\hat{\overline{Y}}(1)) = S^2(1)/I$ because the all $Y_j(1)$ are observed.
      \end{block}
      \begin{block}{ Visualization }	
        \begin{columns}[t,totalwidth=\twocolwid]
          \begin{column}{\onecolwid}
            \begin{figure}[htdp]
              \includegraphics[scale=1]{../Images/FinalLoveplot.pdf}
              \caption*{Love plot - balance checks}
            \end{figure}
          \end{column}
   			\begin{column}{\onecolwid}
             \begin{figure}[htdp]
 	              \includegraphics[scale=0.95]{../Images/intervened.png}
	              \caption*{Interventions and SUTVA}
            \end{figure}
          \end{column}
		\end{columns}
	\end{block}
	\begin{block}{Results}
		\begin{columns}[t,totalwidth=\twocolwid]
	          \begin{column}{\onecolwid}
	            \begin{figure}[htdp]
	              \includegraphics[scale=0.9]{../Images/results.png}
	              %\caption*{Results}
	            \end{figure}
	        \end{column}
\begin{column}{\onecolwid}
	            \begin{figure}[htdp]
	              {\raggedright
	              \begin{minipage}[ht]{0.58\linewidth}
	                \resizebox{25cm}{!}{
					% latex table generated in R 2.14.0 by xtable 1.6-0 package
					% Tue Oct 16 01:14:53 2012
					\begin{tabular}{lcccc}
					  \hline
					\textbf{Region}& number of& Date of first & Regional causal & SD \\
                    & municipalities& intervention&  effect ($\hat{\tau_j}$)& \\ 
					  \hline
					\textbf{Ju\'{a}rez} &  15 & 2009 & 147.54 &20.44 \\ %& 4 
					 \textbf{Nogales}  &   5 & 2008 & 30.60 & 11.17 \\ %& 2
					  \textbf{Tijuana} &   5 & 2008 & 21.18 & 7.34\\ %& 1 
					  \textbf{Acapulco}   &  36 & 2008 & 19.89 & 1.98 \\ %& 18
					\textbf{Sinaloa}  &  29 & 2007 & 11.31 & 4.83 \\ %& 12
					  \textbf{Guadalupe}  &  20 & 2009 & 8.49 & 6.17 \\ %& 8
					   \textbf{Te\'{u}l}&  10 & 2009 & 4.15 & 5.49 \\ % & 10 
					 \textbf{Villa de Cos}  &  22 & 2008 & 3.94 & 8.36 \\ %& 9
					    \textbf{Celaya} &   9 & 2009 & 3.34 & 18.89 \\ % & 15
					   \textbf{Reynosa}  &  24 & 2008 & 0.47 & 5.50 \\ %& 6
					   \textbf{P\'{a}nuco}  &  14 & 2007 & -3.32 & 5.63 \\ %& 5
					   \textbf{Rinc\'{o}n de Romos}  &   7 & 2008 & -10.14 &6.05 \\ %& 11
					   \textbf{Apatzing\'{a}n}  &   9 & 2007 & -25.78 & 4.64 \\%& 16
					\hline
					\hline 
					  $\quad \quad \hat{\tau}$  & 250 & - & 22.42 & 38.10 \\ 
					   \hline
					\end{tabular}}
	              \end{minipage}\hfill
	            }
	              %\caption*{Results}
	            \end{figure}
	          \end{column}
	        \end{columns}
	      \end{block}
    \end{column}


    \begin{column}{\sepwid}\end{column}	


    \begin{column}{\onecolwid}
      % \end{column}
      % \begin{column}{\onecolwid}
\vskip-3ex
      \begin{block}{\url{http://www.processing.org}}
        \begin{figure}[htdp]
          \centering
          \subfigure[2007]{
            \includegraphics[scale=0.43]{2007.png}
          }
          \subfigure[2010]{
            \includegraphics[scale=0.43]{2010.png}
          }
        \end{figure}
	Processing is an open source programming language that allows the creation of dynamic graphics and tables. Due to the size of the data we collected, this tool played a significant role in the display and understanding of the results. The use of dynamic graphics allows:
        \begin{itemize}
          \item The navigation through over 2400 municipalities, and view the homicide rate all within the same window via a map of Mexico.
          \item The presentation of further information, such as cartel presence and homicide rate over time.
          \item The display of matched municipalities and the corresponding causal effects for each region.
        \end{itemize}		
      \end{block}
      \begin{block}{Conclusions}
 	On average, the military interventions result in an increase of the homicide rates. However, the effect varies across the treated regions. The Ju\'{a}rez region is clearly an outlier with an increase of homicide rate of 147 per 100000 inhabitants. 
       Visualizations like this one can help experts to better evaluate the quality of the matches and improve their understanding of the estimates of the causal effects. For example, in this case it is clear that the particularities of the Ju\'{a}rez region play a very important role in the overall average effect. 
      \end{block}
      \begin{block}{Key References }

        \bibliographystyle{plain}
	\vspace{-0.5cm}
        \footnotesize{\begin{thebibliography}{99}
          \bibitem{Abadie} Abadie, Diamond, Hainmueller  \emph{Synthetic Control Methods for Comparative Case Studies} (2009)
          \bibitem{NEXOS} Escalante F, \emph{Homicidios 2008-2009 La muerte tiene permiso} (2011)
          \bibitem{ImbensRubin} Imbens G. \& Rubin D.R., (2012)
          \bibitem{Rubin} Rubin D.R. ,\emph{Matched Sampling for Causal Effects},
\bibitem{Processing} Fry, B. and Reas, C., Processing Library for Visual Arts and Design
            % \bibitem{Valle} Diego Valle visualization
          \end{thebibliography}}        
      \end{block}

	      \begin{block}{Data Sources}
		\centering
	       \small{INEGI, CIDAC, Stratfor, Nexos}
	      \end{block}
    \end{column}
    %\end{columns}
    %\vskip2.5ex
    %\end{column}
    %
    % %\begin{column}{\sepwid}\end{column}	
    % empty spacer column
  \end{columns}
\end{frame}
\end{document}
